\chapter*{Conclusion} % Main chapter title
Pour conclure, j’ai effectué mon stage de deuxième année de cycle d'ingénierie logiciel et intégration des systèmes informatique  en temps que Data Scientist pour l'application mobile Nacharat au sein de l’entreprise SanadTech. Lors de ce stage de 3 mois, j’ai pu mettre en pratique mes connaissances théoriques acquises durant ma formation et aussi ça m'a permis de m'introduire dans une nouvelle route, celle du big data et machine learning.\\[0.5cm]
Pendant cette expérience, j'ai tous d'abord traiter les données,en travailler à bien  saisir son architecture, ce qui m'a permis de les transformer en des matrices numériques, puis après j'ai commencé par aborder le problème de classification depuis le coté de deep learning en utilisant le Réseau neuronal convolutif, ce qui m'a donné une precision en dessus de la moyenne, après j'ai voulu changer l'approche de deep learning, c'est pour cela que cette fois j'ai abordé le problème  par l'une des approches du machine learning, en utilisant le Support vector machine avec un kernel gaussien, il m'a donné un résultat satisfaisant mais pas assez convaincant, et à la fin j'ai repris le premier algorithme développé, et j'ai travaillé à augmenter sa precision en utilisant le réseaux neuronal word2vec dans la phase d'entrer pour minimiser la perte des données et aussi pour données à l'algorithme d'apprendre la sémantique des mots,et tout cela pour avoir une precision bien meilleur.\\[0.5cm]
Pour le Sprint finale, j' y compte rassembler les deux approches en un seul algorithme de classification, l'algorithme va contenir une seule entrée, la première couche va contenir deux parties indépendantes le CNN plus le Word2vec et le SVM , la deuxième couche va ressembler les résultats obtenus par la première couche on un vecteur qu'on va l'analyser par la suite par la méthode de Bayes, pour enfin avoir un résultat avec un taux de precision très élevé.