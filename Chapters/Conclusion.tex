\chapter*{Conclusion} % Main chapter title
Ce stage  m’a permis de découvrir le travail au sein de l’univers de la recherche scientifique et d’approfondir mes connaissances dans le domaine de l’apprentissage statistique, ce qui fût la première des difficultés que j’ai rencontrées et que j’ai dépassées grâce à un travail bibliographique. Je poursuivis ce travail afin d’explorer les différents modèles permettant de traiter de la temporalité en essayant de
suivre une méthodologie claire pour garder un fil conducteur dans cette recherche et aboutir
à la formulation d’un problématique : comment exploiter les relations entre données et
commencer à prendre en compte le temps ? Afin de répondre à cette problématique et grâce
à l’aide des recherches, j’ai mis au point un modèle, de classification par
paquet et implémenté des expériences pour le tester.\\[0.2cm]
En plus des problèmes liés à la tâche, comme la recherche de données par exemple, il fût
délicat pour moi d’adapter les méthodologies enseignées Scrum sur
l’organisation du travail, notamment lors des projets, à une tâche de recherche. Le but de
cette dernière étant plus flou car à plus longue échelle qu’un projet en école d’ingénieur, elle
demande plus de rigueur mais est aussi plus intéressante, d’autant qu’il me sera possible de
continuer la formalisation de ce modèle ou en proposer un nouveau.