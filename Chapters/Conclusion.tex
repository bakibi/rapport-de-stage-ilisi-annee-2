\chapter*{Conclusion \& Perspective} % Main chapter title
\addcontentsline{toc}{chapter}{Conclusion \& Perspective}
Pour conclure, j’ai effectué mon stage de deuxième année de cycle d'ingénierie logiciel et intégration des systèmes informatique  en temps que Data Scientist pour l'application mobile Nacharat au sein de l’entreprise SanadTech. Lors de ce stage de 3 mois, j’ai pu mettre en pratique mes connaissances théoriques acquises durant ma formation et aussi ça m'a permis de m'introduire dans une nouvelle route, celle du big data et machine learning.\\[0.5cm]
Pendant cette expérience, j'ai tous d'abord traité les données,en travaillant sur leur architecture, ce qui m'a permis de les transformer en des matrices numériques. Puis après j'ai commencé par aborder le problème de classification  de type deep learning en utilisant le Réseau neuronal convolutif, ce qui m'a donné une precision en dessus de la moyenne, après j'ai  changé l'approche  du deep learning par celle  du machine learning, en utilisant le Support vector machine avec un kernel gaussien. Il m'a donné un résultat satisfaisant mais pas assez convaincant, et à finalement, j'ai repris le premier algorithme développé, et j'ai travaillé à augmenter sa precision en utilisant le réseaux neuronal word2vec dans la phase d'entrée ce qui permet de minimiser la perte des données et aussi pour données à l'algorithme la capacité d'apprendre la sémantique des mots, afin d'avoir une precision bien meilleur.\\[0.5cm]
Prochainement, en compte rassembler les deux approches en un seul algorithme de classification, ce dernier va contenir une seule entrée,alors que la première couche va contenir deux parties indépendantes le CNN plus le Word2vec et le SVM . La deuxième couche va ressembler les résultats obtenus par la première couche en un vecteur qu'on va l'analyser par la suite par la méthode de Bayes, pour enfin avoir un résultat avec un taux de precision très élevé.