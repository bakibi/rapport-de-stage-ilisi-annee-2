% Introduction

\chapter*{Introduction}
\addcontentsline{toc}{chapter}{Introduction}
Depuis le début de l’informatique, l’homme cherche à communiquer avec les machines. Si les nombreux langages de programmation permettent une forme d’échange entre l’homme et la machine, on aimerait que cette communication se fasse de façon plus naturelle. Pour que cela soit possible, il faut d’abord que la machine “comprenne” ce que l’utilisateur lui dit ensuite qu’elle soit capable de répondre d’une manière compréhensible par l’homme. La discipline derrière ce processus s’appelle le Natural Langage Processing (NLP) ou traitement Automatique du Langage Naturel (TALN) en français. Elle étudie la compréhension, la manipulation et la génération du langage naturel par les machines. Par langage naturel, on entend le langage utilisé par les humains dans leur communication de tous les jours par opposition aux langages artificiels comme les langages de programmation ou les notations mathématiques.\\
C'est dans ce contexte que s'inscrit le projet cadre en sein de l'entreprise SANADTECH, qui envisage d'établir un classificateur de texte à étiquettes multiples pour analyser les titres de son application mobile NACHARAT.\\
NACHARAT est une application mobile qui permet à ses utilisateurs d'être au courant des nouveautés en temps réel, à travers un ensemble de panoplies de news ( hespress, hibapress, ChoufTV, ya bila...etc.).\\ 
Éventuellement, ce document décrit les étapes de la mise en œuvre de ce projet structuré en six chapitres principales, à savoir:

\begin{itemize}
\item La première Chapitre présente la partie management de projet en passant par une présentation rapide des acteurs du projet, méthode agile utilisée, et la planification du projet.
\item Le deuxième chapitre contient la conception globale du projet, en la présente par un ensemble de diagrammes, et descriptions.
\item Le troisième chapitre présente les données traitées, et la façon dont ils ont été manipulées.
\item Le quatrième,cinquième et sixième chapitre  exposent de façon détaillée les différentes étapes de la mise en place du classificateur, de l'analyse globale du projet, vers les différents algorithmes utilisés pour le classificateur.
\end{itemize}